\chapter{``V'' Standard Extension for Vector Operations, Version 0.1}
\label{sec:bits}

This chapter presents a proposal for the RISC-V vector instruction set
extension.  The vector extension supports a configurable vector unit,
to tradeoff the number of architectural vector registers and supported
element widths against available maximum vector length.  The vector
extension is designed to allow the same binary code to work
efficiently across a variety of hardware implementations varying in
physical vector storage capacity and datapath parallelism.

\begin{commentary}
The vector extension is based on the style of vector register
architecture introducted by Seymour Cray in the 1970s, as opposed to
the earlier packed SIMD approach, introduced with the Lincoln Labs
TX-2 in 1957 and now adopted by most other commerical instruction
sets.

The vector instruction set contains many features developed in earlier
research projects, including the Berkeley T0 and VIRAM vector
microprocessors, the MIT Scale vector-thread processor, and the
Berkeley Maven and Hwacha projects.
\end{commentary}

\section{Vector Unit State}

The additional vector unit architectural state consists of 32 vector
data registers ({\tt v0}--{\tt v31}), 8 vector predicate registers
({\tt vp0}-{\tt vp7}), and an XLEN-bit WARL vector length CSR, {\tt
  vl}.  In addition, the current configuration of the vector unit is
held in a set vector configuration CSRs ({\tt vcmaxw}, {\tt vctype},
{\tt vcnpred}), as described below.  The implementation determines an
available {\em maximum vector length} (MVL) for the current
configuration held in the {\tt vcmaxw} and {\tt vcnpred} registers.
There is also a 3-bit fixed-point rounding mode CSR {\tt vxrm}, and a
single-bit fixed-point saturation status CSR {\tt vxsat}.

\begin{table}
  \centering
  \begin{tabular}{|l|c|l|}
    \hline
    CSR name & Number & Base ISA \\
    \hline
    {\tt vl}    & 0x020 & RV32, RV64, RV128 \\
    {\tt vxrm}  & 0x020 & RV32, RV64, RV128 \\
    {\tt vxsat} & 0x020 & RV32, RV64, RV128 \\
    {\tt vcsr}  & 0x020 & RV32, RV64, RV128 \\
    \hline
    {\tt vcnpred} & 0x020 & RV32, RV64, RV128 \\
    \hline
    {\tt vcmaxw}  & 0x020 & RV32, RV64, RV128 \\
    {\tt vcmaxw1} & 0x020 & RV32 \\
    {\tt vcmaxw2} & 0x020 & RV32, RV64 \\
    {\tt vcmaxw3} & 0x020 & RV32 \\
    \hline
    {\tt vctype}  & 0x020 & RV32, RV64, RV128 \\
    {\tt vctype1} & 0x020 & RV32 \\
    {\tt vctype2} & 0x020 & RV32, RV64 \\
    {\tt vctype3} & 0x020 & RV32 \\
    \hline
    {\tt vctypev0}  & 0x020 & RV32, RV64, RV128 \\
      {\tt vctypev1}  & 0x020 & RV32, RV64, RV128 \\
        ... \\
      {\tt vctypev31}  & 0x020 & RV32, RV64, RV128 \\
   \hline
  \end{tabular}
  \caption{Vector extension CSRs.}
  \label{tab:vcsrs}
\end{table}

\section{Element Datatypes and Width}

The datatypes and operations supported by the V extension depend upon
the base scalar ISA and supported extensions, and may include 8-bit,
16-bit, 32-bit, 64-bit, and 128-bit integer and fixed-point data types
(X8, X16, X32, X64, and X128 respectively), and 16-bit, 32-bit,
64-bit, and 128-bit floating-point types (F16, F32, F64, and F128
respectively).  When the V extension is added, it must support the
vector data element types implied by the supported scalar types as
defined by Table~\ref{tab:velemtypes}.  The largest element width
supported:
\[ \mbox{\em ELEN} = max(\mbox{\em XLEN}, \mbox{\em FLEN}) \]

\begin{commentary}
  Compiler support for vectorization is greatly simplified when any
  hardware-supported data types are supported by both scalar and
  vector instructions.
\end{commentary}

\begin{table}
  \centering
\begin{tabular}{|l|l|}
  \hline
  \multicolumn{2}{|c|}{Supported Fixed-Point Widths} \\
  \hline
  RV32I  & X8, X16, X32 \\
  RV64I  & X8, X16, X32, X64 \\
  RV128I & X8, X16, X32, X64, X128 \\
  \hline
  \hline
  \multicolumn{2}{|c|}{Supported Floating-Point Widths} \\
  \hline
  F      & F16, F32 \\
  FD     & F16, F32, F64 \\
  FDQ    & F16, F32, F64, F128 \\
  \hline
\end{tabular}
\caption{Supported data element widths depending on base integer ISA
  and supported floating-point extensions.  Note that supporting a
  given floating-point width mandates support for all narrower
  floating-point widths.}
\label{tab:velemtypes}
\end{table}

Adding the vector extension to any machine with floating-point support
adds support for the IEEE standard half-precision 16-bit
floating-point data type.  This includes a set of scalar
half-precision instructions described in
Section~\ref{sec:scalarhalffloat}.  The scalar half-precision
instructions follow the template for other floating-point precisions,
but using the hitherto unused {\em fmt} field encoding of {\tt 10}.

\begin{commentary}
  We only support scalar half-precision floating-point types as part
  of the vector extension, as the main benefits of half-precision are
  obtained when using vector instructions that amortize per-operation
  control overhead.  Not supporting a separate scalar half-precision
  floating-point extension also reduces the number of standard
  instruction-set variants.
\end{commentary}

\section{Vector Configuration Registers ({\tt vcmaxw}, {\tt
    vctype}, {\tt vcp})}

The vector unit must be configured before use.  Each architectural
vector data register ({\tt v0}--{\tt v31}) is configured with the
maximum number of bits allowed in each element of that vector data
register, or can be disabled to free physical vector storage for other
architectural vector data registers.  The number of available
vector predicate registers can also be set independently.

The available MVL depends on the configuration setting, but MVL must
always have the same value for the same configuration parameters on a
given implementation.  Implementations must provide an MVL of at least
four elements for all supported configuration settings.

Each vector data register's current maximum-width is held in a
separate four-bit field in the {\tt vcmaxw} CSRs, encoded as shown in
Table~\ref{tab:vcmaxw}.

\begin{table}[hbt]
  \centering
  \begin{tabular}{|r|c|}
    \hline
    Width & Encoding \\
    \hline
    Disabled  & 0000 \\
    8         & 1000  \\
    16        & 1001  \\
    32        & 1010  \\
    64        & 1011  \\
    128       & 1100  \\
    \hline
  \end{tabular}
  \caption{Encoding of {\tt vcmaxw} fields. All other values are
    reserved.}
  \label{tab:vcmaxw}
\end{table}

\begin{commentary}
  Several earlier vector machines had the ability to configure
  physical vector register storage into a larger number of short
   vectors or a shorter number of long vectors, in particular the
  Fujitsu VP series~\cite{vp200}.
\end{commentary}

In addition, each vector data register has an associated dynamic type
field that is held in a four-bit field in the {\tt vctype} CSRs,
encoded as shown in Table~\ref{tab:vctype}.  The dynamic type field of
a vector data register is constrained to only hold types that have
equal or lesser width than the value in the corresponding {\tt vcmaxw}
field for that vector data register.  Changes to {\tt vctype} do not
alter MVL.

\begin{table}[hbt]
  \centering
  \begin{tabular}{|l|c|c|}
    \hline
    Type & {\tt vctype} encoding & {\tt vcmaxw} equivalent\\
    \hline
    Disabled & 0000 & 0000 \\
    F16      & 0001 & 1001 \\
    F32      & 0010 & 1010 \\
    F64      & 0011 & 1011 \\
    F128     & 0100 & 1100 \\
    X8       & 1000 & 1000  \\
    X16      & 1001 & 1001  \\
    X32      & 1010 & 1010  \\
    X64      & 1011 & 1011  \\
    X128     & 1100 & 1100  \\
    \hline
  \end{tabular}
  \caption{Encoding of {\tt vctype} fields. The third column shows the
    value that will be saved when writing to {\tt vcmaxw} fields. All
    other values are reserved.}
  \label{tab:vctype}
\end{table}

\begin{commentary}
  Vector data registers have both a maximum element width and a
  current element data type to support vector function calls, where
  the caller does not know the types needed by the callee, as
  described below.
\end{commentary}

To reduce configuration time, writes to a {\tt vcmaxw} field also
write the corresponding {\tt vctype} field.  The {\tt vcmaxw} field
can be written any value taken from the type encoding in
Table~\ref{tab:vctype}, but only the width information as shown in
Table~\ref{tab:vcmaxw} will be recorded in the {\tt vcmaxw} fields
whereas the full type information will be recorded in the
corresponding {\tt vctype} field.

Attempting to write any {\tt vcmaxw} field with a width larger than
that supported by the implementation will raise an illegal instruction
exception.  Implementations are allowed to record a {\tt vcmaxw} value
larger than the value requested.  In particular, an implementation may
choose to hardwire {\tt vcmaxw} fields to the largest supported width.

Attempting to write an unsupported type or a type that requires more
than the current {\tt vcmaxw} width to a {\tt vctype} field will raise
an exception.

Any write to a field in the {\tt vcmaxw} register configures the
vector unit and causes all vector data registers to be zeroed and all
vector predicate registers to be set, and the vector length register
{\tt vl} to be set to the maximum supported vector length.

Any write to a {\tt vctype} field zeros only the associated vector
data register, leaving the other vector unit state undisturbed.
Attempting to write a type needing more bits than the corresponding
{\tt vcmaxw} value to a {\tt vctype} field will raise an illegal
instruction exception.

\begin{commentary}
  Vector registers are zeroed on reconfiguration to prevent security
  holes and to avoid exposing differences between how different
  implementations manage physical vector register storage.

  In-order implementations will probaby use a flag bit per register to
  mux in 0 instead of garbage values on each source until it is
  overwritten.  For in-order machines, partial writes due to
  predication or vector lengths less than MVL complicate this zeroing,
  but these cases can be handled by adopting a hardware
  read-modify-write, adding a zero bit per element, or a trap to
  machine-mode trap handler if first write access after configuration
  is partial.  Out-of-order machines can just point initial rename
  table at physical zero register.
\end{commentary}

%% Can support larger number of architectural vector registers with
%% future extensions.

In RV128, {\tt vcmaxw} is a single CSR holding 32 4-bit width
fields.  Bits $(4N+3)$--$(4N)$ hold the maximum width of vector data
register $N$.  In RV64, the {\tt vcmaxw2} CSR provides access to the
upper 64 bits of {\tt vcmaxw}.  In RV32, the {\tt vcmaxw1} CSR
provides access to bits 63--32 of {\tt vcmaxw}, while {\tt vcmax3} CSR
provides access to bits 127--96.

The {\tt vcnpred} CSR contains a single 4-bit WLRL field giving the
number of enabled architectural predicate registers, between 0 and 8.
Any write to {\tt vcnpred} zeros all vector data registers, sets all
bits in visible vector predicate registers, and sets the vector length
register {\tt vl} to the maximum supported vector length.  Attempting
to write a value larger than 8 to {\tt vcnpred} raises an illegal
instruction exception.

\section{Vector Length}

The active vector length is held in the XLEN-bit WARL vector length
CSR {\tt vl}, which can only hold values between 0 and MVL inclusive.
Any writes to the maximum configuration registers ({\tt vcmaxw} or
{\tt vcnpred}) cause {\tt vl} to be initialized with MVL.  Writes to
{\tt vctype} do not affect {\tt vl}.

The active vector length is usually written with the {\tt setvl}
instruction, which is encoded as a {\tt csrrw} instruction to the {\tt
  vl} CSR number.  The source argument to the {\tt csrrw} is the
requested application vector length (AVL) as an unsigned XLEN-bit
integer. The {\tt setvl} instruction calculates the value to assign to
{\tt vl} according to Table~\ref{tab:vlcalc}.

\begin{table}
  \centering
  \begin{tabular}{|c|c|}
    \hline
    AVL Value & {\tt vl} setting \\
    \hline
    AVL $\geq$ 2\,MVL & MVL \\
    2\,MVL $>$ AVL $>$ MVL & $\lfloor$AVL$/2\rfloor$ \\
    MVL $\geq$ AVL & AVL \\
    \hline
  \end{tabular}
  \caption{Operation of {\tt setvl} instruction to set vector
    length register {\tt vl} based on requested application vector
    length (AVL) and current maximum vector length (MVL).}
  \label{tab:vlcalc}
\end{table}

\begin{commentary}
  The rules for setting the {\tt vl} register help keep vector
  pipelines full over the last two iterations of a stripmined loop.
  Similar rules were previously used in Cray-designed machines~\cite{crayx1asm}.
\end{commentary}

The {\tt vl} register is updated with the minimum of AVL and
MVL, and this value is also returned as the result of the {\tt setvl}
instruction.  Note that unlike a regular {\tt csrrw} instruction, the
value returned is not the original CSR value but the modified value.

\begin{commentary}
  The idea of having implementation-defined vector length dates back
  to at least the IBM 3090 Vector Facility~\cite{ibm370varch}, which
  used a special ``Load Vector Count and Update'' (VLVCU) instruction
  to control stripmine loops.  The {\tt setvl} instruction included
  here is based on the simpler {\tt setvlr} instruction introduced by
  Asanovi\'{c}~\cite{krstephd}.
\end{commentary}

The {\tt setvl} instruction is typically used at the start of every
iteration of a stripmined loop to set the number of vector elements to
operate on in the following loop iteration.  The current MVL can be
obtained by performing a {\tt setvl} with a source argument that has
all bits set (largest unsigned integer).

No element operations are performed for any vector instruction when
{\tt vl}=0.

\begin{figure}[bt]
  \centering
\begin{verbatim}
               # Vector-vector 32-bit add loop.
               # Assume vector unit configured with correct types.
               # a0 holds N
               # a1 holds pointer to result vector
               # a2 holds pointer to first source vector
               # a3 holds pointer to second source vector.
          loop:  setvl t0, a0
                 vld v0, a2      # Load first vector
                 sll t1, t0, 2   # multiply by bytes
                 add a2, t1      # Bump pointer
                 vld v1, a3      # Load second vector
                 add a3, t1      # Bump pointer
                 vadd v0, v1     # Add elements
                 sub a0, t0      # Decrement elements completed
                 vst  v0, a1     # Store result vector
                 add a1, t1      # Bump pointer
                 bnez a0, loop   # Any more?
\end{verbatim}
\caption{Example vector-vector add loop.}
\label{fig:vvadd}
\end{figure}

\section{Rapid Configuration Instructions}

It can take several instructions to set {\tt vcmaxw}, {\tt vctype} and
{\tt vcnpred} to a given configuration.  To accelerate configuring the
vector unit, specialized {\tt vcfg} instructions are added that are
encoded as writes to CSRs with encoded immediate values that set
multiple fields in the {\tt vcmaxw}, {\tt vctype}, and {\tt vncpred}
configuration registers.

The {\tt vcfgd} instruction is encoded as a CSRRW that takes a
register value encoded as shown in Figure~\ref{fig:vdcfg}, and which
returns the corresponding MVL in the destination register.  A
corresponding {\tt vcfgdi} instruction is encoded as a CSRRWI that
takes a 5-bit immediate value to set the configuration, and returns
MVL in the destination register.

\begin{commentary}
  One of the primary uses of {\tt vcfgdi} is to configure the vector
  unit with single-byte element vectors for use in {\tt memcpy} and
  {\tt memset} routines.  A single instruction can configure the
  vector unit for these operation.
\end{commentary}

The {\tt vcfgd} instruction also clears the {\tt vcnpred} register, so
no predicate registers are allocated.

\begin{figure}[hbt]
  \centering
  \begin{tabular}{p{2cm}p{2cm}ccc|c|c|c|c|c|c|c|l}
    \cline{6-12}
     &   &    &      &  & 0 & F64 & F32 & F16 & X32 & X16 & X8 & RV32 \\
    \cline{6-12}
    \multicolumn{1}{c}{} &
    \multicolumn{1}{c}{} &
    \multicolumn{1}{c}{} &
    \multicolumn{1}{c}{} & 
    \multicolumn{1}{c}{} &
    \multicolumn{1}{c}{2} &
    \multicolumn{1}{c}{5} &
    \multicolumn{1}{c}{5} &
    \multicolumn{1}{c}{5} &
    \multicolumn{1}{c}{5} &
    \multicolumn{1}{c}{5} &
    \multicolumn{1}{c}{5} &  \\
    \cline{2-12}
     & \multicolumn{2}{|c|}{0} & \multicolumn{1}{c|}{F128} & \multicolumn{2}{c|}{X64} & F64 & F32 & F16 & X32 & X16 & X8 & RV64 \\
    \cline{2-12}
    \multicolumn{1}{c}{} &
    \multicolumn{2}{c}{24} &
    \multicolumn{1}{c}{5} & 
    \multicolumn{2}{c}{5} &
    \multicolumn{1}{c}{5} &
    \multicolumn{1}{c}{5} &
    \multicolumn{1}{c}{5} &
    \multicolumn{1}{c}{5} &
    \multicolumn{1}{c}{5} &
    \multicolumn{1}{c}{5} &  \\
    \cline{1-12}
    \multicolumn{2}{|c|}{0} & \multicolumn{1}{c|}{X128} &
    \multicolumn{1}{c|}{F128} & \multicolumn{2}{c|}{X64} & F64 & F32 & F16 & X32 & X16 & X8 & RV128 \\
    \cline{1-12}
    \multicolumn{2}{c}{83} &
    \multicolumn{1}{c}{5} &
    \multicolumn{1}{c}{5} & 
    \multicolumn{2}{c}{5} &
    \multicolumn{1}{c}{5} &
    \multicolumn{1}{c}{5} &
    \multicolumn{1}{c}{5} &
    \multicolumn{1}{c}{5} &
    \multicolumn{1}{c}{5} &
    \multicolumn{1}{c}{5} &  \\
  \end{tabular}
  \caption{Format of the {\tt vcfgd} value for different base ISAs,
    holding 5-bit vector register numbers for each supported
    type. Fields must either contain 0 indicating no vector registers
    are allocated for that type, or a vector register number greater
    than all to the right.  All vector register numbers inbetween two
    non-zero fields are allocated to the type with the higher vector
    register number. }
  \label{fig:vdcfg}
\end{figure}

The {\tt vcfgd} value specifies how many vector registers of each
datatype are allocated, and is divided into 5-bit fields, one per
supported datatype.  A value of 0 in a field indicates that no
registers of that type are allocated.  A non-zero value indicates the
highest vector

Each 5-bit field in the {\tt vcfgd} value must contain either zero,
indicating that no vector registers are allocated for that type, or a
vector register number greater than all fields in lower bit positions,
indicating the highest vector register containing the associated type.
This encoding can compactly represent any arbitrary allocation of
vector registers to data types, except that there must be at least two
vector registers ({\tt v0} and {\tt v1}) allocated to the narrowest
required type. An example allocation is shown in
Figure~\ref{fig:vcfgdexample}.

\begin{figure}
  \centering
  \begin{tabular}{|c|c|c|c|c|c|c|}
    \hline
     0 & F64 & F32 & F16 & X32 & X16 & X8 \\
     \hline
     \hline
     0 & 18  & 12  & 0   &  1  &  0  & 0  \\
    \hline
  \end{tabular}
  \\
  \vspace{0.1in}
  \begin{tabular}{|c|c|c|c|}
    \hline
    Vector registers & {\tt vcmaxw} & {\tt vctype} & Type \\
    \hline
    {\tt v31}--{\tt v19} & \tt 0000     & \tt 0000 & Disabled \\
    {\tt v18}--{\tt v13} & \tt 1011     & \tt 0011 & F64 \\
    {\tt v12}--{\tt v2}  & \tt 1010     & \tt 0010 & F32 \\
    {\tt v1}--{\tt v0}   & \tt 1010     & \tt 1010 & X32 \\
    \hline
  \end{tabular}
  \caption{Example use of {\tt vcfgd} value to set configuration.}
  \label{fig:vcfgdexample}
\end{figure}

Separate {\tt vcfgp} and {\tt vcfgpi} instructions are provided, using
the CSRRW and CSRRWI encodings respectively, that write the source
value to the {\tt vcnpred} register and return the new MVL.  These
writes also clear the vector data registers, set all bits in the
allocated predicate registers, and set {\tt vl}=MVL. A {\tt vcfgp} or
{\tt vcfgpi} instruction can be used after a {\tt vcfgd} to complete a
reconfiguration of the vector unit.

If a zero argument is given to {\tt vcgfd} the vector unit will be
unconfigured with no enabled registers, and the value 0 will be
returned for MVL.  Only the configuration registers {\tt vcmaxw} and
{\tt vcnpred} can be accessed in this state, either directly or via
{\tt vcfgd}, {\tt vcfgdi}, {\tt vcfgp}, or {\tt vcfgpi}
instructions. Other vector instructions will raise an illegal
instruction exception.

To quickly change the individual types of a vector register, each
vector data register $n$ has a dedicated CSR address to access its
{\tt vctype} field, named {\tt vctypev}$n$.  The {\tt vcfgt} and {\tt
  vcfgti} instructions are assembler pseudo-instructions for regular
CSRRW and CSRRWI instructions that update the type fields and return
the original value.  The {\tt vcfgti} instruction is typically used to
change to a desired type while recording the previous type in one
instruction, and the {\tt vcfgt} instruction is used to revert back to
the saved type.



%% # integer vector-scalar/scalar-vector operations use low-order bits of
%% # scalar operand.

%% 3130 9 8 7 6 5 4 3 2 120 9 8 7 6 5 4 3 2 110 9 8 7 6 5 4 3 2 1 0
%% +-+-+-+-+-+-+-+-+-+-+-+-+-+-+-+-+-+-+-+-+-+-+-+-+-+-+-+-+-+-+-+-+
%% |   func7     |   rs2   |   rs1   |func3|   rd    | opcode  |1 1|
%% +-+-+-+-+-+-+-+-+-+-+-+-+-+-+-+-+-+-+-+-+-+-+-+-+-+-+-+-+-+-+-+-+
%% +-+-+-+-+-+-+-+-+-+-+-+-+-+-+-+-+-+-+-+-+-+-+-+-+-+-+-+-+-+-+-+-+
%% |   rs3   |fn2|   rs2   |   rs1   |func3|   rd    | opcode  |1 1|
%% +-+-+-+-+-+-+-+-+-+-+-+-+-+-+-+-+-+-+-+-+-+-+-+-+-+-+-+-+-+-+-+-+

%% Uses two reserved opcodes in 32-bit space:
%% VOP  = 10101 11
%% VMEM = 11101 11

%% VOP

%% func3 (xs2, xs1, xf)

%% mostly encodes which scalar values are accessed as with ROCC.

%% xs2  - 1 = reads scalar rs2
%% xs1  - 1 = reads scalar rs1
%% xf   - 0=integer/1=float applies to both


%% Integer arithmetic

%% 3130 9 8 7 6 5 4 3 2 120 9 8 7 6 5 4 3 2 110 9 8 7 6 5 4 3 2 1 0
%% +-+-+-+-+-+-+-+-+-+-+-+-+-+-+-+-+-+-+-+-+-+-+-+-+-+-+-+-+-+-+-+-+
%% |   func7     |   rs2   |   rs1   |func3|   rd    | opcode  |1 1|
%% +-+-+-+-+-+-+-+-+-+-+-+-+-+-+-+-+-+-+-+-+-+-+-+-+-+-+-+-+-+-+-+-+
%% # Integer add/sub
%% # sign-extend smaller source
%% # wraparound overflow into destination
%%  0 0 0 0 0 0 0    vs2       vs1    0 0 0    vd     1 0 1 0 1 1 1  vadd.vv vd, vs2, vs1
%%  0 0 0 0 0 0 0    vs2       rs1    0 1 0    vd     1 0 1 0 1 1 1  vadd.vs vd, vs2, rs1
%%  0 0 0 0 0 0 0    vs2       vs1    0 0 0    vd     1 0 1 0 1 1 1  vsub.vv vd, vs2, vs1
%%  0 0 0 0 0 0 0    vs2       rs1    0 1 0    vd     1 0 1 0 1 1 1  vsub.vs vd, vs2, rs1
%%  0 0 0 0 0 0 0    vs2       rs1    0 1 0    vd     1 0 1 0 1 1 1  vsub.sv vd, rs1, vs2
%% # zero-extend smaller source
%% # wraparound overflow into destination
%%  0 0 0 0 0 0 0    vs2       vs1    0 0 0    vd     1 0 1 0 1 1 1  vaddu.vv vd, vs2, vs1
%%  0 0 0 0 0 0 0    vs2       rs1    0 1 0    vd     1 0 1 0 1 1 1  vaddu.vs vd, vs2, rs1
%%  0 0 0 0 0 0 0    vs2       vs1    0 0 0    vd     1 0 1 0 1 1 1  vsubu.vv vd, vs2, vs1
%%  0 0 0 0 0 0 0    vs2       rs1    0 1 0    vd     1 0 1 0 1 1 1  vsubu.vs vd, vs2, rs1
%%  0 0 0 0 0 0 0    vs2       rs1    0 1 0    vd     1 0 1 0 1 1 1  vsubu.sv vd, rs1, vs2
%% # Shifts use low bits of vsrc2, enough for src1 width
%% # srl/a fills in zero/sign bits in destination
%% # wraparound overflow into destination
%%  0 0 0 0 0 0 0    vs2       vs1    0 0 0    vd     1 0 1 0 1 1 1  vsll.vv vd, vs2, vs1
%%  0 0 0 0 0 0 0    vs2       rs1    0 1 0    vd     1 0 1 0 1 1 1  vsll.vs vd, vs2, rs1
%%  0 0 0 0 0 0 0    vs2       rs1    0 1 0    vd     1 0 1 0 1 1 1  vsll.sv vd, rs1, vs2 (less useful)
%%  0 0 0 0 0 0 0    vs2       vs1    0 0 0    vd     1 0 1 0 1 1 1  vsrl.vv vd, vs2, vs1
%%  0 0 0 0 0 0 0    vs2       rs1    0 1 0    vd     1 0 1 0 1 1 1  vsrl.vs vd, vs2, rs1
%%  0 0 0 0 0 0 0    vs2       rs1    0 1 0    vd     1 0 1 0 1 1 1  vsrl.sv vd, rs1, vs2 (table lookup)
%%  0 0 0 0 0 0 0    vs2       vs1    0 0 0    vd     1 0 1 0 1 1 1  vsra.vv vd, vs2, vs1
%%  0 0 0 0 0 0 0    vs2       rs1    0 1 0    vd     1 0 1 0 1 1 1  vsra.vs vd, vs2, rs1
%%  0 0 0 0 0 0 0    vs2       rs1    0 1 0    vd     1 0 1 0 1 1 1  vsra.vs vd, rs1, vs2 (less useful)
%% # Logical ops
%%  0 0 0 0 0 0 0    vs2       vs1    0 0 0    vd     1 0 1 0 1 1 1  vand.vv vd, vs2, vs1
%%  0 0 0 0 0 0 0    vs2       rs1    0 1 0    vd     1 0 1 0 1 1 1  vand.vs vd, vs2, rs1
%%  0 0 0 0 0 0 0    vs2       vs1    0 0 0    vd     1 0 1 0 1 1 1  vor.vv vd, vs2, vs1
%%  0 0 0 0 0 0 0    vs2       rs1    0 1 0    vd     1 0 1 0 1 1 1  vor.vs vd, vs2, rs1
%%  0 0 0 0 0 0 0    vs2       vs1    0 0 0    vd     1 0 1 0 1 1 1  vxor.vv vd, vs2, vs1
%%  0 0 0 0 0 0 0    vs2       rs1    0 1 0    vd     1 0 1 0 1 1 1  vxor.vs vd, vs2, rs1
%% # Predicate setting (only pd = 0-7 valid)
%% # smaller source is sign-extended
%%  0 0 0 0 0 0 0    vs2       vs1    0 0 0    pd     1 0 1 0 1 1 1  vpeq.vv pd, vs2, vs1
%%  0 0 0 0 0 0 0    vs2       rs1    0 1 0    pd     1 0 1 0 1 1 1  vpeq.vs pd, vs2, rs1
%%  0 0 0 0 0 0 0    vs2       vs1    0 0 0    pd     1 0 1 0 1 1 1  vpne.vv pd, vs2, vs1
%%  0 0 0 0 0 0 0    vs2       rs1    0 1 0    pd     1 0 1 0 1 1 1  vpne.vs pd, vs2, rs1
%%  0 0 0 0 0 0 0    vs2       vs1    0 0 0    pd     1 0 1 0 1 1 1  vplt.vv pd, vs2, vs1
%%  0 0 0 0 0 0 0    vs2       rs1    0 1 0    pd     1 0 1 0 1 1 1  vplt.vs pd, vs2, rs1
%%  0 0 0 0 0 0 0    vs2       rs1    0 1 0    pd     1 0 1 0 1 1 1  vplt.sv pd, rs1, vs2
%% # smaller source is zero-extended (not sure if needed for eq/neq)
%%  0 0 0 0 0 0 0    vs2       vs1    0 0 0    pd     1 0 1 0 1 1 1  vpequ.vv pd, vs2, vs1
%%  0 0 0 0 0 0 0    vs2       rs1    0 1 0    pd     1 0 1 0 1 1 1  vpequ.vs pd, vs2, rs1
%%  0 0 0 0 0 0 0    vs2       vs1    0 0 0    pd     1 0 1 0 1 1 1  vpneu.vv pd, vs2, vs1
%%  0 0 0 0 0 0 0    vs2       rs1    0 1 0    pd     1 0 1 0 1 1 1  vpneu.vs pd, vs2, rs1
%%  0 0 0 0 0 0 0    vs2       vs1    0 0 0    pd     1 0 1 0 1 1 1  vpltu.vv pd, vs2, vs1
%%  0 0 0 0 0 0 0    vs2       rs1    0 1 0    pd     1 0 1 0 1 1 1  vpltu.vs pd, vs2, rs1
%%  0 0 0 0 0 0 0    vs2       rs1    0 1 0    pd     1 0 1 0 1 1 1  vpltu.sv pd, rs1, vs2


%% # Multiply/Divide
%% vvmul vdest, vsrc1, vsrc2  # Signed multiply
%% vsmul vdest, vsrc1, xsrc2  # Signed multiply

%% vvmulh vdest, vsrc1, vsrc2  # Signed multiply
%% vsmulh vdest, vsrc1, xsrc2  # Signed multiply

%% vvmulu vdest, vsrc1, vsrc2  # Unsigned multiply
%% vsmulu vdest, vsrc1, xsrc2  # Unsigned multiply

%% vvmulhu vdest, vsrc1, vsrc2  # Unsigned multiply
%% vsmulhu vdest, vsrc1, xsrc2  # Unsigned multiply

%% vvmulsu vdest, vsrc1, vsrc2  # Signed-unsigned multiply
%% vsmulsu vdest, vsrc1, xsrc2  # Signed-unsigned multiply
%% svmulsu vdest, xsrc1, vsrc2  # Signed-unsigned multiply
%% vvmulhsu vdest, vsrc1, vsrc2  # Signed-unsigned multiply
%% vsmulhsu vdest, vsrc1, xsrc2  # Signed-unsigned multiply
%% svmulhsu vdest, xsrc1, vsrc2  # Signed-unsigned multiply

%% vvdiv vdest, vsrc1, vsrc2
%% vsdiv vdest, vsrc1, xsrc2
%% svdiv vdest, xsrc1, vsrc2

%% vvdivu vdest, vsrc1, vsrc2
%% vsdivu vdest, vsrc1, xsrc2
%% svdivu vdest, xsrc1, vsrc2

%% vvrem vdest, vsrc1, vsrc2
%% vsrem vdest, vsrc1, xsrc2
%% svrem vdest, xsrc1, vsrc2

%% vvremu vdest, vsrc1, vsrc2
%% vsremu vdest, vsrc1, xsrc2
%% svremu vdest, xsrc1, vsrc2

%% # Load/store, size/type given by destination register configuration

%% 3130 9 8 7 6 5 4 3 2 120 9 8 7 6 5 4 3 2 110 9 8 7 6 5 4 3 2 1 0
%% +-+-+-+-+-+-+-+-+-+-+-+-+-+-+-+-+-+-+-+-+-+-+-+-+-+-+-+-+-+-+-+-+
%% |   func7     |   rs2   |   rs1   |func3|   rd    | opcode  |1 1|
%% +-+-+-+-+-+-+-+-+-+-+-+-+-+-+-+-+-+-+-+-+-+-+-+-+-+-+-+-+-+-+-+-+
%% # Unit-stride
%% # option to add post-increment to vld/vst using rs2 a la t0, but can do same with fusion
%%  0 0 0 0 0 0 0 0 0 0 0 0    rs1    0 1 0    vd     1 1 1 0 1 1 1  vld vd, rs1
%%  0 0 0 0 0 0 0 0 0 0 0 0    rs1    0 1 0    vd     1 1 1 0 1 1 1  vst vd, rs1
%% # Constant-stride
%% # Can add segments with immediate field in func7
%%  0 0 0 0 0 0 0    rs2       rs1    1 1 0    vd     1 1 1 0 1 1 1  vlds vd, rs1, rs2
%%  0 0 0 0 0 0 0    rs2       rs1    1 1 0    vd     1 1 1 0 1 1 1  vsts vd, rs1, rs2
%% # Indexed (scatter/gather)
%% # Scalar base + vector offsets
%% # Can add segments with immediate field in func7
%%  0 0 0 0 0 0 0    vs2       rs1    0 1 0    vd     1 1 1 0 1 1 1  vldx vd, rs1, vs2
%%  0 0 0 0 0 0 0    vs2       rs1    0 1 0    vd     1 1 1 0 1 1 1  vstx vd, rs1, vs2
%% # If A extension present:
%% # Vector atomics use vector base address
%% #  t = M[vs2]; M[vs2] = t op vs1; vd = t
%% # must be matching integer 32b (W) or 64b (D) types in vs1 and vd
%%  0 0 0 0 0 0 0    vs2       vs1    0 1 0    vd     1 1 1 0 1 1 1  vamoswap.vv vd, vs2, vs1
%%  0 0 0 0 0 0 0    vs2       rs1    0 1 0    vd     1 1 1 0 1 1 1  vamoswap.vs vd, vs2, rs1
%%  0 0 0 0 0 0 0    vs2       vs1    0 1 0    vd     1 1 1 0 1 1 1  vamoadd.vv vd, vs2, vs1
%%  0 0 0 0 0 0 0    vs2       rs1    0 1 0    vd     1 1 1 0 1 1 1  vamoadd.vs vd, vs2, rs1
%%  0 0 0 0 0 0 0    vs2       vs1    0 1 0    vd     1 1 1 0 1 1 1  vamoand.vv vd, vs2, vs1
%%  0 0 0 0 0 0 0    vs2       rs1    0 1 0    vd     1 1 1 0 1 1 1  vamoand.vs vd, vs2, rs1
%%  0 0 0 0 0 0 0    vs2       vs1    0 1 0    vd     1 1 1 0 1 1 1  vamoor.vv vd, vs2, vs1
%%  0 0 0 0 0 0 0    vs2       rs1    0 1 0    vd     1 1 1 0 1 1 1  vamoor.vs vd, vs2, rs1
%%  0 0 0 0 0 0 0    vs2       vs1    0 1 0    vd     1 1 1 0 1 1 1  vamoxor.vv vd, vs2, vs1
%%  0 0 0 0 0 0 0    vs2       rs1    0 1 0    vd     1 1 1 0 1 1 1  vamoxor.vs vd, vs2, rs1
%%  0 0 0 0 0 0 0    vs2       vs1    0 1 0    vd     1 1 1 0 1 1 1  vamoxor.vv vd, vs2, vs1
%%  0 0 0 0 0 0 0    vs2       rs1    0 1 0    vd     1 1 1 0 1 1 1  vamoxor.vs vd, vs2, rs1
%%  0 0 0 0 0 0 0    vs2       vs1    0 1 0    vd     1 1 1 0 1 1 1  vamomax.vv vd, vs2, vs1
%%  0 0 0 0 0 0 0    vs2       rs1    0 1 0    vd     1 1 1 0 1 1 1  vamomax.vs vd, vs2, rs1
%%  0 0 0 0 0 0 0    vs2       vs1    0 1 0    vd     1 1 1 0 1 1 1  vamomaxu.vv vd, vs2, vs1
%%  0 0 0 0 0 0 0    vs2       rs1    0 1 0    vd     1 1 1 0 1 1 1  vamomaxu.vs vd, vs2, rs1
%%  0 0 0 0 0 0 0    vs2       vs1    0 1 0    vd     1 1 1 0 1 1 1  vamomin.vv vd, vs2, vs1
%%  0 0 0 0 0 0 0    vs2       rs1    0 1 0    vd     1 1 1 0 1 1 1  vamomin.vs vd, vs2, rs1
%%  0 0 0 0 0 0 0    vs2       vs1    0 1 0    vd     1 1 1 0 1 1 1  vamominu.vv vd, vs2, vs1
%%  0 0 0 0 0 0 0    vs2       rs1    0 1 0    vd     1 1 1 0 1 1 1  vamominu.vs vd, vs2, rs1


%% # Memory speculative options.  If permission fault (not just page
%% # fault), then set sticky bit in predicate register vp1 rather than dying.




%% # Example Code
%% ----------------------------------------------------------------------

%% memset(a0=dest, a1=c, a2=len)
%%       csrwi vdcfg, 1    # One vector register of 8b
%%       mv t1, a0         # Copy dest
%%       beqz a1, loop     # Skip scalar move if a1=0 (could drop this instruction)
%%       setvl t0, a2      # Set/find vector length
%%       vmv.vs v0, a1     # Copy scalar a1 to elements of v0
%% loop: setvl t0, a2      # Set/find vector length
%%       vst t1, v0        # Set memory
%%       sub a2, t0        # Decrement count
%%       add t1, t0        # Bump pointer
%%       bnez a2, loop     # Any more?
%% done: vuncfg
%%       j ra


%% # With ai
%% memset(a0=dest, a1=c, a2=len)
%%       csrwi vdcfg, 1    # One vector register of 8b
%%       mv t1, a0         # Copy dest
%%       beqz a1, loop     # Skip scalar move if a1=0 (could drop this instruction)
%%       setvl t0, a2      # Set/find vector length
%%       vmv.vs v0, a1     # Copy scalar a1 to elements of v0
%% loop: setvl t0, a2      # Set/find vector length
%%       vstai t1, v0, t0  # Set memory
%%       sub a2, t0        # Decrement count
%%       bnez a2, loop     # Any more?
%% done: vuncfg
%%       j ra

%% ----------------------------------------------------------------------

%% memcpy(a0=dest, a1=src, a2=len)
%%       csrwi vdcfg, 1    # One vector register of 8b
%%       mv t2, a0         # Copy dest
%% loop: setvl t0, a2      # Set/find vector length
%%       vld v0, a1        # Load vector
%%       add a1, t0        # Bump pointer (can fuse with vld)
%%       sub a2, t0        # Decrement count
%%       vst t2, a0        # Store vector
%%       add t2, t0        # Bump pointer (can fuse with vst)
%%       bnez a2, loop     # Any more?
%% done: vuncfg
%%       j ra

%% # with ldai/stai
%% memcpy(a0=dest, a1=src, a2=len)
%%       csrwi vdcfg, 1    # One vector register of 8b
%%       mv t2, a0         # Copy dest
%% loop: setvl t0, a2      # Set/find vector length
%%       vldai v0, a1, t0  # Load vector
%%       sub a2, t0        # Decrement count
%%       vstai t2, a0, t0  # Store vector
%%       bnez a2, loop     # Any more?
%% done: vuncfg
%%       j ra



